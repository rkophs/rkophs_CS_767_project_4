\documentclass{article}
\usepackage{amsfonts}
\usepackage{amsthm}
\usepackage{amssymb}
\usepackage{hyperref}
\usepackage{algorithm}
\usepackage{algorithmic}
\usepackage{listings}
\usepackage[overload]{empheq}

\newcommand{\Neighbor}{\text{Neighbor}}
\newcommand{\E}{\text{E}}
\newcommand{\PAccept}{\textbf{P}}
\newcommand{\Random}{\textbf{Random}}
\newcommand{\Crossover}{\textbf{Crossover}}
\newcommand{\Mutate}{\textbf{Mutate}}
\newcommand{\Fitness}{\textbf{Fitness}}
\newcommand{\Cost}{\textbf{Cost}}
\newcommand{\Solution}{\textbf{Solution}}
\newcommand{\LeftBisect}{\textbf{leftBisect}}
\newcommand{\If}{\text{ If }}
\newcommand{\Otherwise}{\text{ Otherwise }}
\newcommand{\argmin}{\operatornamewithlimits{\textbf{arg min}}}
\newcommand{\addmedskip}{\addvspace{\medskipamount}}
\newcommand{\addbigskip}{\addvspace{\bigskipamount}}

\setlength{\textheight}{9in}
\setlength{\textwidth}{7.1in}
\setlength{\evensidemargin}{-0.2in}
\setlength{\oddsidemargin}{-0.2in}
\setlength{\headsep}{30pt}
\setlength{\topmargin}{-0.3in}


\begin{document}

\begin{center}
\Large{\textbf{MET CS 767 - Project 4}}
\end{center}
\textbf{Name: Ryan Kophs} \\
\textbf{Date: 12 December 2016}

\section{Tasks completed}
\begin{itemize}
\item Implemented the fuel cell optimization equation presented in [1] using a genetic algorithm heuristic
\item Implemented the fuel cell optimization equation presented in [1] using the heuristic presented in [2].
\item Constructed a web application interface to configure and execute multiple runs of either algorithm implementation.
\end{itemize}

\section{Web Interface}
The web interface may be accessed via public link or by compiling locally. It is recommended to use the public link, as this has the most up to date code.
\subsection{Publicly}
Accessed the application at \url{https://rkophs.github.io/rkophs_CS_767_project_4}
\subsection{Locally}
The web application may also be compiled locally if running locally is preferred (although the public link has the most updated code). To compile and run:

\begin{lstlisting}[language=bash]
  $ cd /path/to/project/
  $ yarn
  $ webpack
  $ python -m SimpleHTTPServer
\end{lstlisting}

Access the application at \url{http://localhost:8000}

\subsection{Settings}
Settings may be expanded by clicking on the settings link. It provides the ability to set all the fuel cell constants, cell parameter bounds, and the observed actual fuel cell parameters. It also provides links to set the parameters specific to the genetic algorithm and the algorithm proposed by [2]. Settings may be supressed by toggling the settings link.
\subsection{Results}
Results are displayed per execution run underneath the settings section. Switch between execution results by clicking the respective run links on the right. A graph shows a series of lines - each being the best solution obtained for a different generation/iteration of the algorithm. The observed actual fuel cell stack is highlighted in green, and the best solution found the algorithm that reduces the sum of squared error between the solution and the observed actual stack is highlighted in red. 

The minimal sum of squared error is displayed above the graph. Below the graph are the results breakdowns per generation/iteration, and the configurations used for the run, defined in settings.

\section{Implementation}
The webapplication is built using React [3] and and Redux [4]. All code pertaining specfically to the fuel cell problem splace and machine learning algorithms resides in \url{/app/machineLearning/}. This directory is broken up as follows:

\begin{itemize}
    \item \url{/app/machineLearning/GeneticAlgorithm.js}: General genetic algorithm that can be applied to any problem space.
    \item \url{/app/machineLearning/GeneticIndividual.js}: A factory method that constructs a single solution to any problem space by accepting $\Crossover, \Mutate, \Fitness, \Cost, \Solution$ call back function in the the problem space.
    \item \url{/app/machineLearning/JPSAlgorithm.js}: General algorithm proposed in [2] that can be applied to any problem space.
    \item \url{/app/machineLearning/FuelCell.js}: The individual calculations for a fuel cell potential given input constants and parameters declared in [1].
    \item \url{/app/machineLearning/FuelCellML.js}: A factory method that constructs a single individual solution pertaining to the fuel cell stack potential. Implements the $\Crossover, \Mutate, \Fitness, \Cost, \Solution$ call backs used by \url{/app/machineLearning/GeneticIndividual.js}. 
\end{itemize}

\section{Inputs and Solution}

\subsection{Inputs}

All inputs are defined as the defaults on the settings page of the application. The are as follows:

Cell Parameter Bounds:
\begin{center}
\begin{tabular}{|c | c | c | c | c | c | c | c|} 
\hline
& $x_1$ & $x_2$ & $x_3$ & $x_4$ & $\lambda$ & $R_c (\Omega)$ & $B (V)$ \\ 
\hline
Lower & -1.19969 & 0.001 & 0.000036 & -0.00026 & 10 & 0.0001 & 0.0136 \\ 
\hline
Upper & -0.8532 & 0.005 & 0.000098 & -0.0000954 & 24 & 0.0008 & 0.5 \\ 
\hline
\end{tabular}
\end{center}

Observed (actual) Fuel Cell Parameters:
\begin{center}
\begin{tabular}{|c | c | c | c | c | c | c | c|} 
\hline
& $x_1$ & $x_2$ & $x_3$ & $x_4$ & $\lambda$ & $R_c (\Omega)$ & $B (V)$ \\ 
\hline
Actual & -0.944957 & 0.00301801 & 0.00007401 & -0.000188 & 23 & 0.0001 & 0.02914489 \\ 
\hline
\end{tabular}
\end{center}

Fuel Cell Constants:
\begin{center}
\begin{tabular}{| c | c | c | c | c | c | c | c | c |} 
\hline
$A (cm^2)$ & $N_s$ & $RH_a$ & $l (cm)$ & $RH_c$ & $\rho_a (atm)$ & $\rho_c (atm)$ & $T (K)$ & $i_{limit,den} (A/cm^2)$ \\ 
\hline
27 & 24 & 1 & 0.0127 & 1 & 2.96077 & 4.93462 & 353.15 & 0.86 \\ 
\hline
\end{tabular}
\end{center}

Genetic Algorithm Parameters:
\begin{center}
\begin{tabular}{| c | c | c | c | c | c | c |} 
\hline
Noise & Max Current $(A)$ & Generations & Child rate & Mutation Rate & Population & PRG Seed \\ 
\hline
0.3 & 20 & 100 & 2 & 0.1 & 400 & 123 \\ 
\hline
\end{tabular}
\end{center}

Algorithm Proposed in [2] Parameters:
\begin{center}
\begin{tabular}{| c | c | c | c | c | c | c | c | c |} 
\hline
Noise & Max Current $(A)$ & Iterations & $\alpha$ & Mutation Rate & Population & PRG Seed & $evalFrac$ & Target Cost \\ 
\hline
0.3 & 20 & 1000 & 0.999 & 0.2 & 400 & 123 & 0.1 & 1.2 \\ 
\hline
\end{tabular}
\end{center}

\subsection{Solution}

With the inputs defined above, the Genetic Algorithm resulted in a sum of squares of 1.2454399. The following shows its resulting current and potential relationships:
\begin{center}
\begin{tabular}{| c | c | c |} 
\hline
Current $(A)$ & Calculated Potential $(V)$ & Observed (Actual) Potential $(V)$\\ 
\hline
1 & 23.455987642743438 & 23.564554482390516 \\ 
2 & 22.266071132995584 & 22.24605976795055 \\ 
3 & 21.533698896349215 & 21.671246947536986\\ 
4 & 20.987287939879614 & 21.1274826128611 \\ 
5 & 20.541583158530862 & 20.229085854654382 \\ 
6 & 20.158468583948707 & 20.43843255959356 \\ 
7 & 19.81746715049399 & 19.23365640902375 \\ 
8 & 19.506201648853413 & 19.267538832399143 \\ 
9 & 19.216518975308425 & 19.66066261240516 \\ 
10 & 18.942660124538627 & 18.867660801177234 \\ 
11 & 18.680293079265184 & 18.656749539340336 \\ 
12 & 18.42594897604348 & 18.36686923581295 \\ 
13 & 18.176656990658813 & 18.36421709805783 \\ 
14 & 17.929673317380036 & 18.05773556826612 \\ 
15 & 17.682238373686456 & 17.562141060717035 \\ 
16 & 17.431303326370443 & 17.21126480052054 \\ 
17 & 17.173144739941858 & 17.11364785799601 \\ 
18 & 16.902710752432355 & 17.205571871654968 \\ 
19 & 16.61232303013411 & 16.62344467263446 \\ 
20 & 16.288644917967673 & 16.006013706044644 \\ 
\hline
\end{tabular}
\end{center}

With the inputs defined above, the algorithm proposed in [2] resulted in a sum of squares of 1.6155517. The following shows its resulting current and potential relationships:
\begin{center}
\begin{tabular}{| c | c | c |} 
\hline
Current $(A)$ & Calculated Potential $(V)$ & Observed (Actual) Potential $(V)$\\ 
\hline
1 & 23.190057959619537 & 23.564554482390516 \\ 
2 & 22.04300667240566 & 22.24605976795055 \\ 
3 & 21.333125053735824 & 21.671246947536986\\ 
4 & 20.800950224227797 & 21.1274826128611 \\ 
5 & 20.365049620298592 & 20.229085854654382 \\ 
6 & 19.989034849618445 & 20.43843255959356 \\ 
7 & 19.653377419970667 & 19.23365640902375 \\ 
8 & 19.34629680813374 & 19.267538832399143 \\ 
9 & 19.060060210369855 & 19.66066261240516 \\ 
10 & 18.789237915618017 & 18.867660801177234 \\ 
11 & 18.52978464625933 & 18.656749539340336 \\ 
12 & 18.27850850774464 & 18.36686923581295 \\ 
13 & 18.032733401523934 & 18.36421709805783 \\ 
14 & 17.790057172527675 & 18.05773556826612 \\ 
15 & 17.548146598597707 & 17.562141060717035 \\ 
16 & 17.304520676517086 & 17.21126480052054 \\ 
17 & 17.05626049663695 & 17.11364785799601 \\ 
18 & 16.799531650260327 & 17.205571871654968 \\ 
19 & 16.52864945927416 & 16.62344467263446 \\ 
20 & 16.23390805690545 & 16.006013706044644 \\ 
\hline
\end{tabular}
\end{center}

\section{Resources}
\begin{enumerate}
    \item Chakraborty, U. K., Abbott, T. E., \& Das, S. K. (2012). PEM fuel cell modeling using differential evolution. Energy, 40(1), 387-399. doi:10.1016/j.energy.2012.01.039
    \item Chakraborty, U. (2012). A new stochastic algorithm for proton exchange membrane fuel cell stack design optimization. Journal of Power Sources, 216, 530-541. doi:10.1016/j.jpowsour.2012.06.040
    \item \url{https://facebook.github.io/react/}
    \item \url{http://redux.js.org/}
    \item \url{https://en.wikipedia.org/wiki/Genetic_algorithm}
    \item \url{https://en.wikipedia.org/wiki/Fitness_proportionate_selection}
\end{enumerate}

\end{document}
